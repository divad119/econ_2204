% Options for packages loaded elsewhere
% Options for packages loaded elsewhere
\PassOptionsToPackage{unicode}{hyperref}
\PassOptionsToPackage{hyphens}{url}
\PassOptionsToPackage{dvipsnames,svgnames,x11names}{xcolor}
%
\documentclass[
  letterpaper,
  DIV=11,
  numbers=noendperiod]{scrartcl}
\usepackage{xcolor}
\usepackage{amsmath,amssymb}
\setcounter{secnumdepth}{-\maxdimen} % remove section numbering
\usepackage{iftex}
\ifPDFTeX
  \usepackage[T1]{fontenc}
  \usepackage[utf8]{inputenc}
  \usepackage{textcomp} % provide euro and other symbols
\else % if luatex or xetex
  \usepackage{unicode-math} % this also loads fontspec
  \defaultfontfeatures{Scale=MatchLowercase}
  \defaultfontfeatures[\rmfamily]{Ligatures=TeX,Scale=1}
\fi
\usepackage{lmodern}
\ifPDFTeX\else
  % xetex/luatex font selection
\fi
% Use upquote if available, for straight quotes in verbatim environments
\IfFileExists{upquote.sty}{\usepackage{upquote}}{}
\IfFileExists{microtype.sty}{% use microtype if available
  \usepackage[]{microtype}
  \UseMicrotypeSet[protrusion]{basicmath} % disable protrusion for tt fonts
}{}
\makeatletter
\@ifundefined{KOMAClassName}{% if non-KOMA class
  \IfFileExists{parskip.sty}{%
    \usepackage{parskip}
  }{% else
    \setlength{\parindent}{0pt}
    \setlength{\parskip}{6pt plus 2pt minus 1pt}}
}{% if KOMA class
  \KOMAoptions{parskip=half}}
\makeatother
% Make \paragraph and \subparagraph free-standing
\makeatletter
\ifx\paragraph\undefined\else
  \let\oldparagraph\paragraph
  \renewcommand{\paragraph}{
    \@ifstar
      \xxxParagraphStar
      \xxxParagraphNoStar
  }
  \newcommand{\xxxParagraphStar}[1]{\oldparagraph*{#1}\mbox{}}
  \newcommand{\xxxParagraphNoStar}[1]{\oldparagraph{#1}\mbox{}}
\fi
\ifx\subparagraph\undefined\else
  \let\oldsubparagraph\subparagraph
  \renewcommand{\subparagraph}{
    \@ifstar
      \xxxSubParagraphStar
      \xxxSubParagraphNoStar
  }
  \newcommand{\xxxSubParagraphStar}[1]{\oldsubparagraph*{#1}\mbox{}}
  \newcommand{\xxxSubParagraphNoStar}[1]{\oldsubparagraph{#1}\mbox{}}
\fi
\makeatother


\usepackage{longtable,booktabs,array}
\usepackage{calc} % for calculating minipage widths
% Correct order of tables after \paragraph or \subparagraph
\usepackage{etoolbox}
\makeatletter
\patchcmd\longtable{\par}{\if@noskipsec\mbox{}\fi\par}{}{}
\makeatother
% Allow footnotes in longtable head/foot
\IfFileExists{footnotehyper.sty}{\usepackage{footnotehyper}}{\usepackage{footnote}}
\makesavenoteenv{longtable}
\usepackage{graphicx}
\makeatletter
\newsavebox\pandoc@box
\newcommand*\pandocbounded[1]{% scales image to fit in text height/width
  \sbox\pandoc@box{#1}%
  \Gscale@div\@tempa{\textheight}{\dimexpr\ht\pandoc@box+\dp\pandoc@box\relax}%
  \Gscale@div\@tempb{\linewidth}{\wd\pandoc@box}%
  \ifdim\@tempb\p@<\@tempa\p@\let\@tempa\@tempb\fi% select the smaller of both
  \ifdim\@tempa\p@<\p@\scalebox{\@tempa}{\usebox\pandoc@box}%
  \else\usebox{\pandoc@box}%
  \fi%
}
% Set default figure placement to htbp
\def\fps@figure{htbp}
\makeatother





\setlength{\emergencystretch}{3em} % prevent overfull lines

\providecommand{\tightlist}{%
  \setlength{\itemsep}{0pt}\setlength{\parskip}{0pt}}



 


\KOMAoption{captions}{tableheading}
\makeatletter
\@ifpackageloaded{caption}{}{\usepackage{caption}}
\AtBeginDocument{%
\ifdefined\contentsname
  \renewcommand*\contentsname{Table of contents}
\else
  \newcommand\contentsname{Table of contents}
\fi
\ifdefined\listfigurename
  \renewcommand*\listfigurename{List of Figures}
\else
  \newcommand\listfigurename{List of Figures}
\fi
\ifdefined\listtablename
  \renewcommand*\listtablename{List of Tables}
\else
  \newcommand\listtablename{List of Tables}
\fi
\ifdefined\figurename
  \renewcommand*\figurename{Figure}
\else
  \newcommand\figurename{Figure}
\fi
\ifdefined\tablename
  \renewcommand*\tablename{Table}
\else
  \newcommand\tablename{Table}
\fi
}
\@ifpackageloaded{float}{}{\usepackage{float}}
\floatstyle{ruled}
\@ifundefined{c@chapter}{\newfloat{codelisting}{h}{lop}}{\newfloat{codelisting}{h}{lop}[chapter]}
\floatname{codelisting}{Listing}
\newcommand*\listoflistings{\listof{codelisting}{List of Listings}}
\makeatother
\makeatletter
\makeatother
\makeatletter
\@ifpackageloaded{caption}{}{\usepackage{caption}}
\@ifpackageloaded{subcaption}{}{\usepackage{subcaption}}
\makeatother
\usepackage{bookmark}
\IfFileExists{xurl.sty}{\usepackage{xurl}}{} % add URL line breaks if available
\urlstyle{same}
\hypersetup{
  pdfauthor={Divad Chaudhary},
  colorlinks=true,
  linkcolor={blue},
  filecolor={Maroon},
  citecolor={Blue},
  urlcolor={Blue},
  pdfcreator={LaTeX via pandoc}}


\title{ECON 2204\\
Quiz 1}
\author{Divad Chaudhary}
\date{February 11, 2026}
\begin{document}
\maketitle


\section{Instructions}\label{instructions}

\begin{itemize}
\tightlist
\item
  Time: 2:30-3:45 PM
\item
  Complete this quiz in this Quarto (.qmd) file.
\item
  Render to PDF and submit both:

  \begin{enumerate}
  \def\labelenumi{\arabic{enumi}.}
  \tightlist
  \item
    the \texttt{.qmd} file, and
  \item
    the rendered output (the \texttt{.pdf} file).
  \end{enumerate}
\item
  Unless told otherwise, write your answers directly under each
  question.
\item
  Some questions ask you to use chunk options so that only results
  appear (not code).
\item
  This exam is closed book. No notes, texts, phones, or other study aids
  are allowed.
\item
  The use of generative AI is strictly prohibited
\item
  You may R's help manual by searching in the Help viewer in RStudio
\end{itemize}

\section{Questions}\label{questions}

\begin{enumerate}
\def\labelenumi{\arabic{enumi}.}
\tightlist
\item
  Getting Started {[}5 Marks{]}

  \begin{enumerate}
  \def\labelenumii{(\alph{enumii})}
  \tightlist
  \item
    Create an R project entitled econ\_2204 and connect it to your
    GitHub account. Make sure you click Create git repository.
  \item
    Within the econ\_2204 directory on your local computer, add a new
    folder called quiz\_1.
  \item
    Add the Quiz 1 files to the quiz\_1 directory.
  \item
    Insert the link to your GitHub repository.
  \end{enumerate}
\item
  Quarto Basics

  \begin{enumerate}
  \def\labelenumii{(\alph{enumii})}
  \item
    In the YAML at the top of this file, replace \texttt{YOUR\ NAME}
    with your name. {[}1 Mark{]}
  \item
    Put the following words in the appropriate font:

    \begin{enumerate}
    \def\labelenumiii{(\roman{enumiii})}
    \tightlist
    \item
      Bold {[}0.5 Marks{]}
    \end{enumerate}

    \begin{itemize}
    \tightlist
    \item
      \textbf{bold}
    \end{itemize}

    \begin{enumerate}
    \def\labelenumiii{(\roman{enumiii})}
    \setcounter{enumiii}{1}
    \tightlist
    \item
      Italics {[}0.5 Marks{]}
    \end{enumerate}

    \begin{itemize}
    \tightlist
    \item
      \emph{italics}
    \end{itemize}

    \begin{enumerate}
    \def\labelenumiii{(\roman{enumiii})}
    \setcounter{enumiii}{2}
    \tightlist
    \item
      Code {[}0.5 Marks{]}
    \end{enumerate}

    \begin{itemize}
    \tightlist
    \item
      \texttt{code}
    \end{itemize}
  \end{enumerate}
\item
  Write the following equations using LaTeX math syntax so that they
  render properly. Write them using display math. {[}2 Marks each{]}
\end{enumerate}

\begin{quote}
\begin{enumerate}
\def\labelenumi{(\alph{enumi})}
\tightlist
\item
  The simple linear regression model: \[
  Y_i = \beta_0 + \beta_1 X_i + u_i
  \]
\end{enumerate}
\end{quote}

\begin{quote}
\begin{enumerate}
\def\labelenumi{(\alph{enumi})}
\setcounter{enumi}{1}
\tightlist
\item
  The sample mean: \[
  \bar{X} = \frac{1}{n} \sum_{i=1}^{n} X_i
  \]
\end{enumerate}

\begin{enumerate}
\def\labelenumi{\arabic{enumi}.}
\setcounter{enumi}{2}
\tightlist
\item
  Insert an image using Markdown image syntax. {[}5 Marks{]}
\end{enumerate}

\begin{itemize}
\tightlist
\item
  Use the image in the uw-logo-centre-stack-black.png file, but it must
  render
\item
  Add a caption that reads: Figure 1: University of Winnipeg Logo.
  \pandocbounded{\includegraphics[keepaspectratio]{uw-logo-centre-stack-black.png}}
\end{itemize}

\begin{enumerate}
\def\labelenumi{\arabic{enumi}.}
\setcounter{enumi}{3}
\tightlist
\item
  Insert an R chunk and add an Image Using R. Use the
  \texttt{echo:\ false} execution option, so that the code does not show
  in the PDF.
\end{enumerate}

\begin{enumerate}
\def\labelenumi{(\alph{enumi})}
\tightlist
\item
  Generate a variable \(x=(1,2,3,4,5,6,7,8,9,10)\) {[}1 Mark{]}
\item
  Generate a variable \(y= 2x+5\) {[}1 Mark{]}
\item
  Create a simple plot with \(x\) on the \(x\)-axix and \(y\) on the
  \(y\)-axis using the \texttt{plot()} function.
\item
  Add a figure caption ``Simple Plot of \(X\) Versus \(Y\)'' using the
  quarto execution command \texttt{fig-cap}.
\item
  Add the label \texttt{fig-scatterplot} using the \texttt{label}
  execution command
\item
  Reference the plot in a sentence below the plot
\end{enumerate}
\end{quote}

\begin{quote}
\begin{enumerate}
\def\labelenumi{\arabic{enumi}.}
\setcounter{enumi}{4}
\tightlist
\item
  \{r\}
\end{enumerate}

\begin{itemize}
\tightlist
\item
  \#\textbar{} echo: false
\item
  \#\textbar{} fig-cap: ``Simple Ploy of X Versus Y''
\item
  \#\textbar{} label: fig-scatterplot
\item
  x \textless- c(1:10)
\item
  y \textless- 2*x+ 5 plot (x, y)
\item
  As shown in Figure \textbf{?@fig-scatterplot}, Y increases linearly
  with X.
\end{itemize}

\begin{enumerate}
\def\labelenumi{(\alph{enumi})}
\tightlist
\item
  Create a data frame using \texttt{data.frame()} called
  \texttt{students} with columns \texttt{name} and \texttt{grade} with
  the following rows:
\end{enumerate}
\end{quote}

\begin{longtable}[]{@{}ll@{}}
\toprule\noalign{}
name & grade \\
\midrule\noalign{}
\endhead
\bottomrule\noalign{}
\endlastfoot
Ana & 82 \\
Ben & 75 \\
Cara & 91 \\
Dan & 68 \\
\end{longtable}

\begin{quote}
We want the code to print in the PDF, so set \texttt{echo:\ true}.
Compute and print the average grade. \{r\} \#\textbar{} echo: true
students \textless- data.frame( - name = c(``Ana'', ``Ben'', ``Cara'',
``Dan'') - grade = c(82, 75, 91, 68) ) students - average\_grade
\textless- mean(students\$grade) - average\_grade
\end{quote}

\begin{enumerate}
\def\labelenumi{\arabic{enumi}.}
\setcounter{enumi}{5}
\tightlist
\item
  Commit the finished quiz to your GitHub profile {[}1 Mark{]}
\end{enumerate}




\end{document}
